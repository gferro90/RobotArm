\chapter*{Ringraziamenti}
\addcontentsline{toc}{chapter}{Ringraziamenti}

Se lo si desidera, inserire qui un breve elenco di ringraziamenti riguardo la tesi.\\

Non superare possibilmente la lunghezza di una pagina!


\chapter*{Introduzione}
\addcontentsline{toc}{chapter}{Introduzione}

Il capitolo introduttivo \`e generalmente lungo tre pagine (almeno due). 
Una buona introduzione pu\`o essere preparata secondo il seguente schema caratterizzante tre blocchi consecutivi: 
\begin{enumerate}
\item {\em Introduzione generale all'ambito in cui si colloca la tesi} (pi\`u o meno partendo da ``caro amico"). Ad esempio: ``La robotica nasce dall'esigenza di sostituire l'uomo in quei lavori che... " eccetera.

\item {\em Collocazione della tesi nell'ambito generale sopra descritto}. Ad esempio: ``Questo lavoro di tesi si colloca nel contesto dell'automazione domestica. In particolare, con riferimento a quanto sopra accennato, l'esigenza di .... ''.

\item {\em Descrizione schematica della struttura della relazione} (un paragrafo o poco pi\`u). Ad esempio: ``La tesi \`e strutturata come segue: nel Capitolo~\ref{cap:primo} viene discussa una ..., 

\end{enumerate}

